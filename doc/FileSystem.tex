

\documentclass{article}

\usepackage{fullpage}

\begin{document}

\section*{Introduction}

In order to support multiple updates to a file at any point in a file we
developed a file system which will work with the operating systems filesystem.
Our filesystem supports chunking of data to enable anywhere inserts.
This is different from a write which is destructive.
An insert is when we want to place some data between two points without losing any of the current data.
In current file systems this happens by taking the current data and moving it down and then writing the new data in the open spot.
By using chunking this minimizes the amount of work to do but has some overhead on seeking to a point in a file.

Our current plans are to support creating and deleting files.
We are going to support basic file operations such as read and write.
Additionally we will support remove and insert which take advantage of chunking.


\section*{Layout}

Our file system will be stored to a single file.
In later versions we might work on storing it in multiple files.
In our current file layout we have a single header which contains information about the file system.
This includes things such as how many files, the first block which is free, and so on.
After the header comes the first block which is assigned to the data structure used to manage the files.
This will contain information pertaining to files stored, such as the name and first block they are stored in.

\subsection*{Header layout}

In the header layout we begin with the signature which is just the hex string {\it DEADBEEF}.
This is followed by multiple 64-bit values.

\begin{itemize}
    \item Signature
    \item Size of file on disk
    \item Number of blocks on disk
    \item Number of files
    \item Size of file meta-data
    \item First block of the file metadata
    \item Last block of the file metadata
    \item Number of blocks for the file meta-data
    \item Amount of used bytes for file meta-data
    \item Allocated bytes for file meta-data
\end{itemize}

\section*{Issues}

When writing to a file there are issues to consider.
For instance how do we know when to grow the file.
When we need to grow the file we can check the position in the file in terms of allocated disk space.
We can also save the total amount of bytes allocated on disk.
Using these two values and amount of data we can calculate if we will need more disk space.

The second issue is when writing.  We can write the amount of bytes on each block contiguously and then
find the point where we need to reconnect and move the blocks we have to the end.

\section*{Use cases}

In this section we talk about typical use cases

\subsection*{Creating the file system}

Assumptions are that no file exists.
We do not consider the header area when calculating values.
We begin with attributes related to how much is allocated then followed by attributes related to usage.

\begin{itemize}
    \item   Create the file
    \item   Grow the file to the default size of 2KB
        \item    Move to beginning of the file
        \begin{itemize}
            % Signature
            \item   Write file signature (0xDEADBEAF)
            % Begin storage meta-data
            \item   Write file size allocated (1KB)
            \item   Write number of blocks allocated (1)
            \item   Write file size used (0);
            \item   Write number of blocks used (0)
            % Begin free list meta-data
            \item   Write number of blocks on free list (0)
            \item   Write first free list block (0)
            % Begin file-system meta-data
            \item   Write amount of bytes allocated for file system (BlockSize) 
            \item   Write number of blocks allocated to file system (1)
            \item   Write amount of bytes used by file system (0)
            \item   Write amount of blocks used by file system (0)
            \item   Write number of files created (0)
            \item   Write first block of file-system metadata (1) 
            \item   Write last block of file-system metadata (1) 
        \end{itemize}
    \item   Move to first block in file-system where meta-data is stored
        \begin{itemize}
            \item   Write previous block (0)
            \item   Write next block (0)
            \item   Write length (0)
        \end{itemize}
\end{itemize}


\end{document}
