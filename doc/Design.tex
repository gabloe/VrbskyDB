\documentclass{article}
\usepackage{graphicx}
\usepackage{textcomp, xspace}
\usepackage{fullpage}
\newcommand\la{\textlangle\xspace}
\newcommand\ra{\textrangle\xspace}
\begin{document}

\title{NoSQL Database Design}
\author{Gabriel Loewen \and Jeffrey Robinson}

\maketitle

\section{Description}
We are implementing a document-oriented database management system named VrbskyDB.  In this database management system we store documents as JSON objects.  A JSON object is a collection of key/value pairs where a value can be a string, a number, a boolean, an embedded object, or an array.  When a document is inserted into a database we assign a universally unique identifier (UUID) to the document.  Documents are stored in a virtual filesystem that allows for file creation, file writing, and file reading.  When data is written to a file the virtual filesystem automatically chunks the data into equal sized segments.  The chunking process allows data to grow or shrink without having to extensively manipulate the filesystem structure.  Both linear hash tables and B-tree data structures are built on top of the virtual filesystem, and are used to store metadata and indexes, respectively.  The grammar of VrbskyDB shares a lot of similarities with SQL, but does not allow join operations.

\section{Query Grammar}
The grammar of VrbskyDB shares some similarities with SQL.  We have named the query languge VQL for ``Vrbsky Query Language''.  VQL supports inserting documents, deleting documents, adding data to a document, removing data from a document, and creating indexes.

\subsection{Insert}
Insert one or more documents into a project.  If the specified project does not exist then the project will automatically be created.
\begin{center}
INSERT INTO \la PROJECT NAME \ra WITH \la JSON \ra \\
Or \\
INSERT INTO \la PROJECT NAME \ra WITH [ \la JSON \ra, \la JSON \ra, ... ]
\end{center}

\subsection{Update}
Update one or more fields in a project.
\begin{center}
UPDATE \la PROJECT NAME \ra WITH \la JSON \ra \\
Or \\
UPDATE \la PROJECT NAME \ra WITH \la JSON \ra WHERE \la JSON \ra \\
\end{center}

\subsection{Delete}
Delete one or more fields in a project.
\begin{center}
UPDATE \la PROJECT NAME \ra WITH \la JSON \ra \\
Or \\
UPDATE \la PROJECT NAME \ra WITH \la JSON \ra WHERE \la JSON \ra \\
\end{center}

\subsection{Create}
Create an index on one or more fields.
\begin{center}
CREATE INDEX ON [\la FIELD \ra, \la FIELD \ra, ...]
\end{center}

\subsection{Select}
Search for data in a document with some specified criteria.
\begin{center}
SELECT \la FIELD \ra, ... , \la FIELD \ra IN \la PROEJCT NAME \ra \\
Or
SELECT \la FIELD \ra, ... , \la FIELD \ra IN \la PROEJCT NAME \ra WHERE \la JSON \ra
\end{center}

\section{Storage}

For VrbskyDB we implemented a rudamentary storage engine which mimics a file system.
The engine includes additional operations crucial for databses that native filesystems do not support.
We include an insert and a remove method which allows for arbitrary inserts and removes anywhere in a file.
This has the benefit of removing costly copies to shift data to make room.

These operations are possible due to the data-structure format we used for our files.
Each file is broken into blocks which can hold a fixed amount of data.
When a file is written to we allocate enough blocks for the data.
If at some point we need to perform an insert into the middle of the file we allocate
a new chain of blocks for the new data and relink our file in the proper order.
For removes we perform a similar operation but instead of adding blocks we remove them and 
place them on a special file which represents the free list of blocks.

In addition to the storage engine we also supply a File, FileReader, and FileWriter class which
match the traditional layout of file I/O APIs.

\section{Indexes}

Indexes are planned to be stored in a B-tree data structure.  User's will be able to create an index on any field and
when a query is executed that includes indexed fields the query will use the index for optimized lookups.  The B-tree
will be built on top of the virtual filesystem and be treated like a normal file.

\end{document}
